\section{Summary and Conclusion}
\label{summa_conc}

% \textcolor{red}{
% It would be better what each chapter has the same volume of information, i.e., numbers of pages have to be close.
% However, I know it is not easy to write much information at the conclusion.
% But, You need to write 2 or 3 pages at least.
% For example, please write \\
% - Why did you choose this research \\
% - What was the problem and motivated you to conduct the research \\
% - What types of proposal was required \\
% - How did you overcome the problems in terms of technologies \\
% - How you could confirm effectiveness of the proposal \\
% - and so on \\
% }
Making robots that follow people is one of the most exciting uses of robotics. 
Long-term development can be done from there to provide other uses. At the moment, 
there are few studies on employing robots to track a single object, and these studies 
have mixed results.
The bulk of problems with monitoring people or objects nowadays, in particular, 
use cameras to collect data. It is simple to lose track of things due to the camera's 
still constrained information-collection range and limitations in many complex 
environmental situations. I decided to choose HFR system development as my thesis topic 
as a result of these factors. My study for a graduation project specifically researches 
for the employment of a 3D LiDAR user-following robot system.\\

There are several challenges in the study process because there have not been many studies targeting at 
utilizing 3D LiDAR to identify and track individuals. In contrast to utilizing a 
camera, 3D LiDAR finds fewer characteristics in RGB or RGB-D pictures, making it 
more difficult to identify humans. We had trouble locating and choosing 
characteristics that can be employed in the robot system that follows people 
comparing to using camera in the development of the point-cloud 
user recognition algorithm since not all characteristics that are calculated using 
collected point-clouds are useful. Because that feature does not yet exude 
quiet, some algorithm will lengthen calculation times and some will produce noise during the detection. 
The outcomes of training and testing are also poor since there is no environment 
and training data. We have not been able to test in scenarios with more complexity. 
Even if there are still many obstacles to overcome, we continue to work on 
finishing the research since it is a promising and novel area that requires 
further study in order to be effectively applied to human life.\\

The HFR system with 3D LiDAR is proposed in the thesis, with the 3 key components 
being SVM (online learning) for human tracking and detection, human filter based velocity as well as 
PID control for guiding the robot to the user's position.
We based on earlier research for the human identification and tracking section, 
which uses online learning and a tracker to be able to spot individuals moving. 
In order to increase the accuracy of persons detection, we also introduced a new 
feature which is the volume feature to the SVM model.\\

We have added RANSAC, which removes the point-cloud as a plane, and a boundary filter, 
which is the ROI around the robot, before the human identification section in 
order to increase the processing performance for the online learning in a certain time 
period. In specific, only everything within 10 meters of the robot should be taken into account and we
remove any extraneous objects from the ROI area. Both actions can save processing time and 
eliminate noise. We tested the HFR pipeline in 3 different
environments (Gazebo environment, university's corridor, and testing hall) and this
pipeline performed well enough in all testing cases.\\



% The robot can still find the person even when the robot misclassifies other objects
% during the following process.
% However, there are some weak points in this pipeline that needs to solve in the future:

% \begin{enumerate}
%     \item Robots can only follow the human position without avoiding obstacles
%     \item If the environment has multiple people, especially when people are standing
%           close to each other as groups, the robot can not distinguish which person to follow

% \end{enumerate}

Our study still has some limitations because of the limited research time 
and the lack of expertise and understanding in robotics. Robots can only follow 
people at the moment and they are also unable to go around barriers. In other words, 
robots can only follow the human position without avoiding obstacles and if the environment has multiple people, especially when people are standing
close to each other as groups, the robot can not distinguish which person to follow.
The robot's movement is still fairly sluggish, and it does not react strongly 
to objects that move quickly, particularly in areas with plenty of people and 
items arranged side by side. If they are too close together, single items in 
online learning are difficult to distinguish. 

\newpage

\section{Future Work}
\label{future_work}

Since the HFR system employing 3D LiDAR still has numerous limitations, more study 
into this area may be beneficial. By integrating 3D LiDAR with the camera, 
for instance, it is easier to recognize several individuals standing near to 
one another or people standing next to another item. LiDAR offers precise 3D 
geometry, whereas cameras record additional contextual data 
\cite{fusionlidarandcamera}. In short, we want to fuse other sensors to add more necessary features and better deal with challenging situations such as groups
mentioning in section \ref{summa_conc}.\\

With regards to the issue of navigation and control, the research still only 
uses a basic PID algorithm to move the robot to a person's position. 
To avoid colliding with, harming, or hurting the surrounding environment or the 
robot itself, the robot must be able to avoid obstacles in the fully autonomous 
problem. As a result, the intergration of algorithms such as A*, D*, RRT, 
and others \cite{lspathplanning} should be researched in order to improve and enhance the 
HFR system in addition to obstacle avoidance. In other words, research works require combining obstacles avoidance to make the robot follow the human more efficiently.\\

In the realm of control, PID is still a reliable and straightforward control 
algorithm. The study's findings, however, show that PID is still not particularly 
adept at tracking individuals. The future research can utilize MPC instead of PID 
to follow the path from the robot to the location of the human without the robot 
following straight to the position of the person as in the current study since 
MPC has numerous notable benefits over PID \cite{comparepidmpc}.

% Fuse other sensors to add more necessary features to better deal with challenging situations such as groups

